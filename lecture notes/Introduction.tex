%%%%%%%%%%%%%%%%%%%%%%%%%%%%%%%%%%%%%%%%%
% Beamer Presentation
% LaTeX Template
% Version 1.0 (10/11/12)
%
% This template has been downloaded from:
% http://www.LaTeXTemplates.com
%
% License:
% CC BY-NC-SA 3.0 (http://creativecommons.org/licenses/by-nc-sa/3.0/)
%
%%%%%%%%%%%%%%%%%%%%%%%%%%%%%%%%%%%%%%%%%

%----------------------------------------------------------------------------------------
%	PACKAGES AND THEMES
%----------------------------------------------------------------------------------------

\documentclass{beamer}

\mode<presentation> {

% The Beamer class comes with a number of default slide themes
% which change the colors and layouts of slides. Below this is a list
% of all the themes, uncomment each in turn to see what they look like.

%\usetheme{default}
%\usetheme{AnnArbor}
%\usetheme{Antibes}
%\usetheme{Bergen}
%\usetheme{Berkeley}
%\usetheme{Berlin}
%\usetheme{Boadilla}
%\usetheme{CambridgeUS}
%\usetheme{Copenhagen}
%\usetheme{Darmstadt}
%\usetheme{Dresden}
%\usetheme{Frankfurt}
%\usetheme{Goettingen}
%\usetheme{Hannover}
%\usetheme{Ilmenau}
%\usetheme{JuanLesPins}
%\usetheme{Luebeck}
\usetheme{Madrid}
\usepackage{nameref}
\usepackage{multicol}
\usepackage{pifont}
\usepackage[T5]{fontenc}

\usepackage[utf8]{inputenc}
%\usetheme{Malmoe}
%\usetheme{Marburg}
%\usetheme{Montpellier}
%\usetheme{PaloAlto}
%\usetheme{Pittsburgh}
%\usetheme{Rochester}
%\usetheme{Singapore}
%\usetheme{Szeged}
%\usetheme{Warsaw}

% As well as themes, the Beamer class has a number of color themes
% for any slide theme. Uncomment each of these in turn to see how it
% changes the colors of your current slide theme.

%\usecolortheme{albatross}
%\usecolortheme{beaver}
%\usecolortheme{beetle}
%\usecolortheme{crane}
%\usecolortheme{dolphin}
%\usecolortheme{dove}
%\usecolortheme{fly}
%\usecolortheme{lily}
%\usecolortheme{orchid}
%\usecolortheme{rose}
%\usecolortheme{seagull}
%\usecolortheme{seahorse}
%\usecolortheme{whale}
%\usecolortheme{wolverine}
%\setbeamertemplate{footline} % To remove the footer line in all slides uncomment this line
%\setbeamertemplate{footline}[page number] % To replace the footer line in all slides with a simple slide count uncomment this line

%\setbeamertemplate{navigation symbols}{} % To remove the navigation symbols from the bottom of all slides uncomment this line
}
\usepackage{amsmath,amssymb,latexsym}
\usepackage{graphicx} % Allows including images
\usepackage{booktabs} % Allows the use of \toprule, \midrule and \bottomrule in tables
\setbeamertemplate{caption}[numbered]

%----------------------------------------------------------------------------------------
%	TITLE PAGE
%----------------------------------------------------------------------------------------

\title[Python cho quản trị hệ thống và mạng]{Python cho quản trị hệ thống và mạng } % The short title appears at the bottom of every slide, the full title is only on the title page

\author{Vu Duc Ly} % Your name
%\institute[UCLA] % Your institution as it will appear on the bottom of every slide, may be shorthand to save space

\date{\today} % Date, can be changed to a custom date

\begin{document}


\begin{frame}
\titlepage % Print the title page as the first slide
\hyperlink{intro}{\beamerbutton{Start}}
\end{frame}

\begin{frame}[label=intro]
\frametitle{Giới Thiệu Khóa Học} % Table of contents slide, comment this block out to remove it
\tableofcontents % Throughout your presentation, if you choose to use \section{} and \subsection{} commands, these will automatically be printed on this slide as an overview of your presentation
\end{frame}

%----------------------------------------------------------------------------------------
%	PRESENTATION SLIDES
%----------------------------------------------------------------------------------------

%------------------------------------------------
\section{Mục Tiêu .} 
\begin{frame}[label=muctieu]
\frametitle{Mục Tiêu}
Sau khóa học, học viên có thể
\begin{itemize}
\item Sử dụng python để tự động hóa các công việc quản trị mạng và hệ thống.
\item Sử dụng các phần mềm mã nguồn mở phục vụ cho mục đích quản trị.
\item Xây dựng các plugins cho các phần mềm mã nguồn mở.
\end{itemize} 


\end{frame}

\section{Chương Trình Học.} 
\begin{frame}[label=sylabus]
\frametitle{Chương Trình Học}

Khóa học diễn ra trong 5 tuần với các nội dung như sau. 

\begin{tabular}{ |p{4cm}||p{3cm}||p{3cm}|  }
 \hline
 \multicolumn{3}{|c|}{Chương trình học} \\
 \hline
 Tuần 1& Tuần 2 & Tuần 3 \\
 \hline
  $\bullet$ Thiết lập môi trường python \newline
  $\bullet$ Cài đặt gói (package) thư viện \newline      
  $\bullet$ Cú pháp căn bản \newline
  $\bullet$ Các kiểu biến và tính toán \newline
  $\bullet$ Cấu trúc quyết định và vòng lặp \newline
  $\bullet$ Cấu trúc dữ liệu cơ bản  \newline
  $\bullet$ Sử dụng hàm   \newline
   & $\bullet$ Thu thập thông tin hệ thống: CPU, memory, disk \newline
   $\bullet$ Quản lý các tiến trình (process) \newline $\bullet$ Giao tiếp với firewall \newline   & $\bullet$ Phân tích log của các dịch vụ \newline $\bullet$ Làm việc với các hệ thống cơ sở dữ liệu \newline \\
 \hline
\end{tabular}


\end{frame}

\begin{frame}[label=syllabus]
\frametitle{Chương Trình Học (tiếp theo)}

\begin{tabular}{ |p{4cm}||p{4cm}||  }
 \hline
 \multicolumn{2}{|c|}{Chương trình học} \\
 \hline
 Tuần 4 & Tuần 5  \\
 \hline
$\bullet$ Thu thập và phân tích thông tin từ các thiết bị mạng router, switch \newline
   & $\bullet$ Sử dụng và tích hợp các mã nguồn mở   \newline  $\bullet$ Final project  \newline   \\
 \hline
\end{tabular}


\end{frame}

%------------------------------------------------



\begin{frame}
\Huge{\centerline{Thank you}}
\end{frame}

%----------------------------------------------------------------------------------------

\end{document}