%%%%%%%%%%%%%%%%%%%%%%%%%%%%%%%%%%%%%%%%%
% Beamer Presentation
% LaTeX Template
% Version 1.0 (10/11/12)
%
% This template has been downloaded from:
% http://www.LaTeXTemplates.com
%
% License:
% CC BY-NC-SA 3.0 (http://creativecommons.org/licenses/by-nc-sa/3.0/)
%
%%%%%%%%%%%%%%%%%%%%%%%%%%%%%%%%%%%%%%%%%

%----------------------------------------------------------------------------------------
%	PACKAGES AND THEMES
%----------------------------------------------------------------------------------------

\documentclass{beamer}

\mode<presentation> {

% The Beamer class comes with a number of default slide themes
% which change the colors and layouts of slides. Below this is a list
% of all the themes, uncomment each in turn to see what they look like.

%\usetheme{default}
%\usetheme{AnnArbor}
%\usetheme{Antibes}
%\usetheme{Bergen}
%\usetheme{Berkeley}
%\usetheme{Berlin}
%\usetheme{Boadilla}
%\usetheme{CambridgeUS}
%\usetheme{Copenhagen}
%\usetheme{Darmstadt}
%\usetheme{Dresden}
%\usetheme{Frankfurt}
%\usetheme{Goettingen}
%\usetheme{Hannover}
%\usetheme{Ilmenau}
%\usetheme{JuanLesPins}
%\usetheme{Luebeck}
\usetheme{Madrid}
\usepackage{nameref}
\usepackage{multicol}
\usepackage{pifont}
\usepackage[T5]{fontenc}
\usepackage{listings}
\usepackage[utf8]{inputenc}
%\usetheme{Malmoe}
%\usetheme{Marburg}
%\usetheme{Montpellier}
%\usetheme{PaloAlto}
%\usetheme{Pittsburgh}
%\usetheme{Rochester}
%\usetheme{Singapore}
%\usetheme{Szeged}
%\usetheme{Warsaw}

% As well as themes, the Beamer class has a number of color themes
% for any slide theme. Uncomment each of these in turn to see how it
% changes the colors of your current slide theme.

%\usecolortheme{albatross}
%\usecolortheme{beaver}
%\usecolortheme{beetle}
%\usecolortheme{crane}
%\usecolortheme{dolphin}
%\usecolortheme{dove}
%\usecolortheme{fly}
%\usecolortheme{lily}
%\usecolortheme{orchid}
%\usecolortheme{rose}
%\usecolortheme{seagull}
%\usecolortheme{seahorse}
%\usecolortheme{whale}
%\usecolortheme{wolverine}
%\setbeamertemplate{footline} % To remove the footer line in all slides uncomment this line
%\setbeamertemplate{footline}[page number] % To replace the footer line in all slides with a simple slide count uncomment this line

%\setbeamertemplate{navigation symbols}{} % To remove the navigation symbols from the bottom of all slides uncomment this line
}
\usepackage{amsmath,amssymb,latexsym}
\usepackage{graphicx} % Allows including images
\usepackage{booktabs} % Allows the use of \toprule, \midrule and \bottomrule in tables
\setbeamertemplate{caption}[numbered]

%----------------------------------------------------------------------------------------
%	TITLE PAGE
%----------------------------------------------------------------------------------------

\title[Giao tiếp với các tiến trình, xử lý lỗi ]{Giao tiếp với các tiến trình, xử lý lỗi } % The short title appears at the bottom of every slide, the full title is only on the title page

\author{Vũ Đức Lý} % Your name
%\institute[UCLA] % Your institution as it will appear on the bottom of every slide, may be shorthand to save space

\date{\today} % Date, can be changed to a custom date

\begin{document}


\begin{frame}
\titlepage % Print the title page as the first slide
\hyperlink{intro}{\beamerbutton{Start}}
\end{frame}

\begin{frame}[label=intro]
\frametitle{Nội dung} % Table of contents slide, comment this block out to remove it
\tableofcontents % Throughout your presentation, if you choose to use \section{} and \subsection{} commands, these will automatically be printed on this slide as an overview of your presentation
\end{frame}

%----------------------------------------------------------------------------------------
%	PRESENTATION SLIDES
%----------------------------------------------------------------------------------------

%------------------------------------------------
\section{Giao tiếp với các tiến trình} 
\begin{frame}[label=subprocess]
\frametitle{Giao tiếp với các tiến trình}
Khởi động và giao tiếp với các tiến trình khác. 

$\bullet$ Sử dụng subprocess
\begin{itemize}
\item Chạy một tiến trình khác: $subprocess.call()$
\item Kiểm tra việc gọi tiến trình có thành công: $subprocess.check\_call()$
\item Lấy output trả về: $subprocess.check\_output()$
\item Chạy nhiều command liên tiếp: $subprocess.check\_output('commands here', shell = True)$
\end{itemize} 
\hyperlink{intro}{\beamerbutton{Back}}
\hyperlink{moitruong}{\beamerbutton{Next}}
\end{frame}

\section{Xử lý lỗi}
\begin{frame}[label=error]
\frametitle{Xử lý lỗi}
$\bullet$ Bắt lỗi trong các trường hợp input không đúng. Sử dụng try-catch
\begin{itemize}
\item Câu lệnh cần được kiểm tra bắt lỗi nằm trong try
\item Lỗi được xử lý trong except
\end{itemize}

\hyperlink{teptin}{\beamerbutton{Back}}
\end{frame}

\begin{frame}[label=error]
\frametitle{Các lỗi phổ biến}
$\bullet$ Các kiểu lỗi phổ biến thường gặp khi viết chương trình
\begin{itemize}
\item Các lỗi thuộc về cú pháp
\item Các ngoại lệ
\end{itemize}
\end{frame}

\section{Thao tác trực tiếp với Pipes}
\begin{frame}[label=pipe]
\frametitle{Thao tác trực tiếp với Pipes}
Sử dụng Popen trực tiếp giúp ta điều khiển các các command được chạy và xử lý input và output.  

\begin{itemize}
\item Giao tiếp một chiều với một tiến trình
\item Giao tiếp hai chiều với một tiến trình
\end{itemize}
\hyperlink{teptin}{\beamerbutton{Back}}
\end{frame}





%----------------------------------------------------------------------------------------

\end{document}