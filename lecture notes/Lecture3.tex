%%%%%%%%%%%%%%%%%%%%%%%%%%%%%%%%%%%%%%%%%
% Beamer Presentation
% LaTeX Template
% Version 1.0 (10/11/12)
%
% This template has been downloaded from:
% http://www.LaTeXTemplates.com
%
% License:
% CC BY-NC-SA 3.0 (http://creativecommons.org/licenses/by-nc-sa/3.0/)
%
%%%%%%%%%%%%%%%%%%%%%%%%%%%%%%%%%%%%%%%%%

%----------------------------------------------------------------------------------------
%	PACKAGES AND THEMES
%----------------------------------------------------------------------------------------

\documentclass{beamer}

\mode<presentation> {

% The Beamer class comes with a number of default slide themes
% which change the colors and layouts of slides. Below this is a list
% of all the themes, uncomment each in turn to see what they look like.

%\usetheme{default}
%\usetheme{AnnArbor}
%\usetheme{Antibes}
%\usetheme{Bergen}
%\usetheme{Berkeley}
%\usetheme{Berlin}
%\usetheme{Boadilla}
%\usetheme{CambridgeUS}
%\usetheme{Copenhagen}
%\usetheme{Darmstadt}
%\usetheme{Dresden}
%\usetheme{Frankfurt}
%\usetheme{Goettingen}
%\usetheme{Hannover}
%\usetheme{Ilmenau}
%\usetheme{JuanLesPins}
%\usetheme{Luebeck}
\usetheme{Madrid}
\usepackage{nameref}
\usepackage{multicol}
\usepackage{pifont}
\usepackage[T5]{fontenc}
\usepackage{listings}
\usepackage[utf8]{inputenc}
%\usetheme{Malmoe}
%\usetheme{Marburg}
%\usetheme{Montpellier}
%\usetheme{PaloAlto}
%\usetheme{Pittsburgh}
%\usetheme{Rochester}
%\usetheme{Singapore}
%\usetheme{Szeged}
%\usetheme{Warsaw}

% As well as themes, the Beamer class has a number of color themes
% for any slide theme. Uncomment each of these in turn to see how it
% changes the colors of your current slide theme.

%\usecolortheme{albatross}
%\usecolortheme{beaver}
%\usecolortheme{beetle}
%\usecolortheme{crane}
%\usecolortheme{dolphin}
%\usecolortheme{dove}
%\usecolortheme{fly}
%\usecolortheme{lily}
%\usecolortheme{orchid}
%\usecolortheme{rose}
%\usecolortheme{seagull}
%\usecolortheme{seahorse}
%\usecolortheme{whale}
%\usecolortheme{wolverine}
%\setbeamertemplate{footline} % To remove the footer line in all slides uncomment this line
%\setbeamertemplate{footline}[page number] % To replace the footer line in all slides with a simple slide count uncomment this line

%\setbeamertemplate{navigation symbols}{} % To remove the navigation symbols from the bottom of all slides uncomment this line
}
\usepackage{amsmath,amssymb,latexsym}
\usepackage{graphicx} % Allows including images
\usepackage{booktabs} % Allows the use of \toprule, \midrule and \bottomrule in tables
\setbeamertemplate{caption}[numbered]

%----------------------------------------------------------------------------------------
%	TITLE PAGE
%----------------------------------------------------------------------------------------

\title[Giao tiếp với mạng và cơ sở dữ liệu ]{Giao tiếp với mạng, web và cơ sở dữ liệu } % The short title appears at the bottom of every slide, the full title is only on the title page

\author{Vũ Đức Lý} % Your name
%\institute[UCLA] % Your institution as it will appear on the bottom of every slide, may be shorthand to save space

\date{\today} % Date, can be changed to a custom date

\begin{document}


\begin{frame}
\titlepage % Print the title page as the first slide
\hyperlink{intro}{\beamerbutton{Start}}
\end{frame}

\begin{frame}[label=intro]
\frametitle{Nội dung} % Table of contents slide, comment this block out to remove it
\tableofcontents % Throughout your presentation, if you choose to use \section{} and \subsection{} commands, these will automatically be printed on this slide as an overview of your presentation
\end{frame}

%----------------------------------------------------------------------------------------
%	PRESENTATION SLIDES
%----------------------------------------------------------------------------------------

%------------------------------------------------
\section{Kiểu dữ liệu từ điển (dictionary)} 
\begin{frame}[label=dict]
\frametitle{Kiểu dữ liệu từ điển (dictionary)}
Sử dụng kiểu từ điển khi không quan tâm đến thứ tự của các thành phần trong nó. Biến từ điển thường có hai thành phần: khóa (\textit{key}) và giá trị (\textit{value})

\begin{example}
$process\_dict = \{"pid": 321, "name": "http"\}$
\end{example} 
\hyperlink{intro}{\beamerbutton{Back}}
\hyperlink{moitruong}{\beamerbutton{Next}}
\end{frame}

\begin{frame}[label=dictoper]
\frametitle{Các thao tác trên từ điển}
$\bullet$Lấy giá tri của key. 
\begin{example}
$process\_dict["pid"]$
\end{example}
$\bullet$ Thay đổi giá trị của key, thêm key.
\begin{example}
$process\_dict["pid"] = 123$ 
$process\_dict["start_time"] = 12$
\end{example}
$\bullet$ Xóa một key
\begin{example}
$del  process\_dict["name"]$
\end{example}
\hyperlink{teptin}{\beamerbutton{Back}}
\end{frame}

\begin{frame}[label=vidunmap]
\frametitle{Case study: Thu thập thông tin về mạng cục bộ}
Sử python-nmap để thu thập thông tin về mạng. 

$\bullet$ Lấy thông tin về scan
\begin{example}
$nm.scaninfo()$
\end{example}
$\bullet$ Lấy ra trạng thái port 
\begin{example}
$nm['127.0.0.1']['tcp'][80]['state'] $
\end{example}
$\bullet$ Lấy ra tất cả ports dựa trên giao thức (tcp, ip,...) 
\begin{example}
$nm['127.0.0.1'].all\_tcp() $
\end{example}

\hyperlink{teptin}{\beamerbutton{Back}}
\end{frame}

\section{Sử dụng cơ sở dữ liệu - MongoDB}
\begin{frame}[label=cosodulieu]
\frametitle{Sử dụng cơ sở dữ liệu - MongoDB}
$\bullet$ MongoDB là một phần mềm cơ sở dữ liêu mã nguồn mở được thiết kế cho việc dễ dàng phát triển và mở rộng. MongoDB hỗ trợ Python.
\begin{itemize}
\item Khởi tạo kết nối tới MongoDB, tạo cơ sở dữ liệu
\item chèn dữ liêu vào MongoDB
\item Tìm kiếm dữ liệu
\item Truy vấn dữ liệu
\end{itemize}

\hyperlink{teptin}{\beamerbutton{Back}}
\end{frame}

\section{Lưu trữ thông tin về mạng vào cơ sở dữ liệu}
\begin{frame}[label=tonghop]
\frametitle{Lưu trữ thông tin về mạng vào cơ sở dữ liệu}
$\bullet$ Công việc quản trị hệ thống yêu cầu thu thập thông tin và kiểm tra hệ thống hằng ngày và đưa ra các báo cáo (reports).
\begin{itemize}
\item Sử dụng python nmap hoặc psutil để lấy thông tin hệ thống.
\item Lưu thông tin hệ thống theo ngày
\item Thiết kế giao diện để truy vấn các thông tin
\end{itemize}

\hyperlink{teptin}{\beamerbutton{Back}}
\end{frame}




%----------------------------------------------------------------------------------------

\end{document}