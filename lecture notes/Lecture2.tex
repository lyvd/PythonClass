%%%%%%%%%%%%%%%%%%%%%%%%%%%%%%%%%%%%%%%%%
% Beamer Presentation
% LaTeX Template
% Version 1.0 (10/11/12)
%
% This template has been downloaded from:
% http://www.LaTeXTemplates.com
%
% License:
% CC BY-NC-SA 3.0 (http://creativecommons.org/licenses/by-nc-sa/3.0/)
%
%%%%%%%%%%%%%%%%%%%%%%%%%%%%%%%%%%%%%%%%%

%----------------------------------------------------------------------------------------
%	PACKAGES AND THEMES
%----------------------------------------------------------------------------------------

\documentclass{beamer}

\mode<presentation> {

% The Beamer class comes with a number of default slide themes
% which change the colors and layouts of slides. Below this is a list
% of all the themes, uncomment each in turn to see what they look like.

%\usetheme{default}
%\usetheme{AnnArbor}
%\usetheme{Antibes}
%\usetheme{Bergen}
%\usetheme{Berkeley}
%\usetheme{Berlin}
%\usetheme{Boadilla}
%\usetheme{CambridgeUS}
%\usetheme{Copenhagen}
%\usetheme{Darmstadt}
%\usetheme{Dresden}
%\usetheme{Frankfurt}
%\usetheme{Goettingen}
%\usetheme{Hannover}
%\usetheme{Ilmenau}
%\usetheme{JuanLesPins}
%\usetheme{Luebeck}
\usetheme{Madrid}
\usepackage{nameref}
\usepackage{multicol}
\usepackage{pifont}
\usepackage[T5]{fontenc}
\usepackage{listings}
\usepackage[utf8]{inputenc}
%\usetheme{Malmoe}
%\usetheme{Marburg}
%\usetheme{Montpellier}
%\usetheme{PaloAlto}
%\usetheme{Pittsburgh}
%\usetheme{Rochester}
%\usetheme{Singapore}
%\usetheme{Szeged}
%\usetheme{Warsaw}

% As well as themes, the Beamer class has a number of color themes
% for any slide theme. Uncomment each of these in turn to see how it
% changes the colors of your current slide theme.

%\usecolortheme{albatross}
%\usecolortheme{beaver}
%\usecolortheme{beetle}
%\usecolortheme{crane}
%\usecolortheme{dolphin}
%\usecolortheme{dove}
%\usecolortheme{fly}
%\usecolortheme{lily}
%\usecolortheme{orchid}
%\usecolortheme{rose}
%\usecolortheme{seagull}
%\usecolortheme{seahorse}
%\usecolortheme{whale}
%\usecolortheme{wolverine}
%\setbeamertemplate{footline} % To remove the footer line in all slides uncomment this line
%\setbeamertemplate{footline}[page number] % To replace the footer line in all slides with a simple slide count uncomment this line

%\setbeamertemplate{navigation symbols}{} % To remove the navigation symbols from the bottom of all slides uncomment this line
}
\usepackage{amsmath,amssymb,latexsym}
\usepackage{graphicx} % Allows including images
\usepackage{booktabs} % Allows the use of \toprule, \midrule and \bottomrule in tables
\setbeamertemplate{caption}[numbered]

%----------------------------------------------------------------------------------------
%	TITLE PAGE
%----------------------------------------------------------------------------------------

\title[Thao tác hệ thống file, tiến trình với Python ]{Thao tác các tệp tin, tiến trình với Python } % The short title appears at the bottom of every slide, the full title is only on the title page

\author{Vũ Đức Lý} % Your name
%\institute[UCLA] % Your institution as it will appear on the bottom of every slide, may be shorthand to save space

\date{\today} % Date, can be changed to a custom date

\begin{document}


\begin{frame}
\titlepage % Print the title page as the first slide
\hyperlink{intro}{\beamerbutton{Start}}
\end{frame}

\begin{frame}[label=intro]
\frametitle{Nội dung} % Table of contents slide, comment this block out to remove it
\tableofcontents % Throughout your presentation, if you choose to use \section{} and \subsection{} commands, these will automatically be printed on this slide as an overview of your presentation
\end{frame}

%----------------------------------------------------------------------------------------
%	PRESENTATION SLIDES
%----------------------------------------------------------------------------------------

%------------------------------------------------
\section{Thao tác với hệ thống tệp tin (File System)} 
\begin{frame}[label=teptin]
\frametitle{Thao tác với hệ thống tệp tin (File System)}
Thư viện của Python cung cấp các công cụ trợ giúp
\begin{itemize}
\item Phân tách (parsing) các tên tệp tin \hyperlink{lamviecteptin}{\beamerbutton{$\unrhd$}}
\item Xây dựng các đường dẫn tới tệp tin \hyperlink{xaydungduongdan}{\beamerbutton{$\unrhd$}}.
\item Kiểm tra nội dung của tệp tin.\hyperlink{kiemtranoidung}{\beamerbutton{$\unrhd$}}
\item Xây dựng một script tự động đầu tiên.\hyperlink{scriptdautien}{\beamerbutton{$\unrhd$}}
\end{itemize} 
\hyperlink{intro}{\beamerbutton{Back}}
\hyperlink{moitruong}{\beamerbutton{Next}}
\end{frame}

\begin{frame}[label=lamviecteptin]
\frametitle{Làm việc với các tệp tin với \textit{os.path}}
\textit{os.path} được dùng để phân tách các đường dẫn. 
$\bullet$ Lấy tên file
\begin{example}
$ps.path.split(path)$ hoặc
$os.path.basename(path)$
\end{example}
$\bullet$ Lấy tên thư mục chứa file
\begin{example}
$ps.path.split(path)$ hoặc
$os.path.dirname(path)$
\end{example}
$\bullet$ Lấy phần mở rộng (extension), ví dụ: \textit{.txt, .pdf,...}
\begin{example}
os.path.splitext(path)
\end{example}
\hyperlink{teptin}{\beamerbutton{Back}}
\end{frame}

\begin{frame}[label=xaydungduongdan]
\frametitle{Xây dựng đường dẫn tới tệp tin/thư mục}
Sử dụng hàm \textit{join()} để kết hợp các strings thành một đường dẫn. 

$\bullet$ Xây dựng đường dẫn tới một tệp tin
\begin{example}
my\_path = ['/','home', 'lyvd', 'Documents', 'secret.txt']\newline
new\_path = os.path.join(*my\_path)
\end{example}
$\bullet$ Tự động mở rộng đường dẫn 
\begin{example}
os.path.expanduser(lookup)
\end{example}
$\bullet$ Chuẩn hóa các đường dẫn 
\begin{example}
os.path.normpath(wrong\_path)
\end{example}
\hyperlink{teptin}{\beamerbutton{Back}}
\end{frame}

\begin{frame}[label=kiemtranoidung]
\frametitle{Kiểm tra nội dung của tệp tin/thư mục}
$\bullet$ Kiểm tra đường dẫn có tồn tại, nếu có thì nó chỉ tới một tệp tin hay thư mục
\begin{example}
os.path.exists(path)\newline
os.path.isfile(path)\newline
os.path.isdir(path)
\end{example}
$\bullet$ Lặp qua các tệp tin và thư mục  
\begin{example}
os.walk(path)
\end{example}

\hyperlink{teptin}{\beamerbutton{Back}}
\end{frame}

\begin{frame}[label=scriptdautien]
\frametitle{Xây dựng một script tự động đầu tiên}
$\bullet$ Viết một script có các tính năng sau:
\begin{itemize}
\item Liệt kê tất các người dùng trong một nhóm cho trước.
\item Tạo một thư mục làm việc cho mỗi người dùng (nếu có thư mục trùng tên thì xóa đi)
\end{itemize}


\hyperlink{teptin}{\beamerbutton{Back}}
\end{frame}

\section{Thu thập thông tin về tiến trình, bộ nhớ với psutil.} 
\begin{frame}[label=tientrinh]
\frametitle{Thu thập thông tin về tiến trình, bộ nhớ với psutil}

psutil là một thư viện cho phép lấy các thông tin về các tiến trình đang chạy và khả năng sử dụng hệ thống (CPU, memory, đĩa cứng, và mạng)
\begin{itemize}
\item Thông tin về CPU.
\item Thông tin bộ nhớ
\item Thông tin về đĩa 
\item Thông tin về mạng
\item Các thông tin khác
\end{itemize}
 
\hyperlink{intro}{\beamerbutton{Back}}
\hyperlink{cuphap}{\beamerbutton{Next}}
\end{frame}


%----------------------------------------------------------------------------------------

\end{document}