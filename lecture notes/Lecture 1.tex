%%%%%%%%%%%%%%%%%%%%%%%%%%%%%%%%%%%%%%%%%
% Beamer Presentation
% LaTeX Template
% Version 1.0 (10/11/12)
%
% This template has been downloaded from:
% http://www.LaTeXTemplates.com
%
% License:
% CC BY-NC-SA 3.0 (http://creativecommons.org/licenses/by-nc-sa/3.0/)
%
%%%%%%%%%%%%%%%%%%%%%%%%%%%%%%%%%%%%%%%%%

%----------------------------------------------------------------------------------------
%	PACKAGES AND THEMES
%----------------------------------------------------------------------------------------

\documentclass{beamer}

\mode<presentation> {

% The Beamer class comes with a number of default slide themes
% which change the colors and layouts of slides. Below this is a list
% of all the themes, uncomment each in turn to see what they look like.

%\usetheme{default}
%\usetheme{AnnArbor}
%\usetheme{Antibes}
%\usetheme{Bergen}
%\usetheme{Berkeley}
%\usetheme{Berlin}
%\usetheme{Boadilla}
%\usetheme{CambridgeUS}
%\usetheme{Copenhagen}
%\usetheme{Darmstadt}
%\usetheme{Dresden}
%\usetheme{Frankfurt}
%\usetheme{Goettingen}
%\usetheme{Hannover}
%\usetheme{Ilmenau}
%\usetheme{JuanLesPins}
%\usetheme{Luebeck}
\usetheme{Madrid}
\usepackage{nameref}
\usepackage{multicol}
\usepackage{pifont}
%\usetheme{Malmoe}
%\usetheme{Marburg}
%\usetheme{Montpellier}
%\usetheme{PaloAlto}
%\usetheme{Pittsburgh}
%\usetheme{Rochester}
%\usetheme{Singapore}
%\usetheme{Szeged}
%\usetheme{Warsaw}

% As well as themes, the Beamer class has a number of color themes
% for any slide theme. Uncomment each of these in turn to see how it
% changes the colors of your current slide theme.

%\usecolortheme{albatross}
%\usecolortheme{beaver}
%\usecolortheme{beetle}
%\usecolortheme{crane}
%\usecolortheme{dolphin}
%\usecolortheme{dove}
%\usecolortheme{fly}
%\usecolortheme{lily}
%\usecolortheme{orchid}
%\usecolortheme{rose}
%\usecolortheme{seagull}
%\usecolortheme{seahorse}
%\usecolortheme{whale}
%\usecolortheme{wolverine}
%\setbeamertemplate{footline} % To remove the footer line in all slides uncomment this line
%\setbeamertemplate{footline}[page number] % To replace the footer line in all slides with a simple slide count uncomment this line

%\setbeamertemplate{navigation symbols}{} % To remove the navigation symbols from the bottom of all slides uncomment this line
}
\usepackage{amsmath,amssymb,latexsym}
\usepackage{graphicx} % Allows including images
\usepackage{booktabs} % Allows the use of \toprule, \midrule and \bottomrule in tables
\setbeamertemplate{caption}[numbered]

%----------------------------------------------------------------------------------------
%	TITLE PAGE
%----------------------------------------------------------------------------------------

\title[Weekly Report]{Weekly Report } % The short title appears at the bottom of every slide, the full title is only on the title page

\author{Vu Duc Ly} % Your name
%\institute[UCLA] % Your institution as it will appear on the bottom of every slide, may be shorthand to save space

\date{\today} % Date, can be changed to a custom date

\begin{document}


\begin{frame}
\titlepage % Print the title page as the first slide
\hyperlink{currentwork}{\beamerbutton{Start}}
\end{frame}

\begin{frame}[label=currentwork]
\frametitle{Current Works} % Table of contents slide, comment this block out to remove it
\tableofcontents % Throughout your presentation, if you choose to use \section{} and \subsection{} commands, these will automatically be printed on this slide as an overview of your presentation
\end{frame}

%----------------------------------------------------------------------------------------
%	PRESENTATION SLIDES
%----------------------------------------------------------------------------------------

%------------------------------------------------
\section{Feature Engineering.} 
\begin{frame}[label=fe]
\frametitle{Feature Engineering.}
Before using the instruction sequences as inputs to RNNs, we need to convert the data to sequences of numerical feature vectors. 

\begin{figure}[!ht]
	\centering
	\includegraphics[scale=0.52]{vector_mapping.png}
	\caption{A typical basic block}
	\label{fig:vector_mapping}
\end{figure}
\hyperlink{currentwork}{\beamerbutton{Back}}
\hyperlink{model}{\beamerbutton{Next}}
\end{frame}

\section{Malware Classification Model.} 
\begin{frame}[label=model]
\frametitle{Malware Classification Model}

Our malware classification process is displayed in 

\begin{figure}[!ht]
\centering
\includegraphics[scale=0.4]{system_model.pdf}
\caption{Overview of our malware classification system }
\label{fig:system_model}
\end{figure}

 
\hyperlink{fe}{\beamerbutton{Back}}
\hyperlink{plan}{\beamerbutton{Next}}
\end{frame}




\section{Next Plan}
\begin{frame}[label=plan]
\frametitle{My Plan}
I will 
\begin{itemize}
	\item Collect malware (from Virusshare) and extract malware's instruction sequences 

\end{itemize} 

\hyperlink{currentwork}{\beamerbutton{Back}}

\end{frame}

%------------------------------------------------



\begin{frame}
\Huge{\centerline{Thank you}}
\end{frame}

%----------------------------------------------------------------------------------------

\end{document}